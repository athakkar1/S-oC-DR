\documentclass[conference]{IEEEtran}
\IEEEoverridecommandlockouts
% The preceding line is only needed to identify funding in the first footnote. If that is unneeded, please comment it out.
%Template version as of 6/27/2024

\usepackage{cite}
\usepackage{amsmath,amssymb,amsfonts}
\usepackage{algorithmic}
\usepackage{graphicx}
\usepackage{textcomp}
\usepackage{xcolor}
\def\BibTeX{{\rm B\kern-.05em{\sc i\kern-.025em b}\kern-.08em
    T\kern-.1667em\lower.7ex\hbox{E}\kern-.125emX}}
\begin{document}

\title{S(oC)DR: Interfacing Between a System-on-Chip and Software Defined Radio\\
}

\author{\IEEEauthorblockN{Ajay Thakkar}
\IEEEauthorblockA{\textit{Electrical and Computer Engineering} \\
\textit{Stevens Institute of Technology}\\
Hoboken, New Jersey \\
athakka5@stevens.edu \\
Advised by Professor Bernard Yett}
}

\maketitle

\begin{abstract}
Placeholder
\end{abstract}

\section{Introduction}
Many modern spectrum tasks require portability and computational power to solve complex signal processing tasks in dynamic environments.
Applications like remote sensing and tactical communications require self sufficient RF systems that can operate on the edge while employing
modern spectrum tasks such as spectrum sensing and signal classification. Creating such devices usually requires
custom hardware and software that are tailored to the application, which is both a costly and non-reusable solution. Technology such as
Software Defined Radios (SDR) and System-on-Chips (SoC) can be leveraged in these applications to form fast, reprogrammable, portable, and cost effective systems.

SDRs are flexible RF systems that allow for digital reconfigurability to change its operational parameters while running.
Their novelty lies in the interface between a General-Purpose Processor (GPP) and an RF front end, allowing the user to digitally control the front end's
operating frequency, gain, sampling rate, etc. This GPP also allows for a personal computer to communicate with the system to send commands and retrieve return
data for further processing. Software such as GNU Radio and MATLAB include packages to communicate with popular SDRs as well as signal processing and visualization
blocks that can be tailored to a wide variety of RF tasks. This SDR ecosystem has been used extensively in modern research for prototyping RF systems,
like Hu et al. utilizing the Pluto SDR platform as well as MATLAB's deep learning toolbox to implement a Convolutional Neural Network (CNN) 
for signal classification of the Pluto SDR's return signal. However, since many of these studies use a personal computer for their data processing which have a large form factor 
and utilize a lot of power, these systems are not viable for field testing or deployment. 

To that end, many of these same signal processing applications such as CNNs have been
implemented on programmable logic platforms like SoCs. SoCs contain both Programmable Logic (PL) and a Processing System (PS), essentially 
integrating both FPGA fabric and general purpose processors on the same chip and allowing them to work in tandem. Amin et al. were able to implement a CNN for signal classification
on programmable logic, achieving an accuracy of 99.98\% at 84139 frames per second with only 4.4W of power consumption, showing how aforementioned SDR research can utilize
SoCs to lower power consumption and increase portability while still retaining the same processing power as a personal computer. This paper endeavors to create an interface
between commercial SoCs and SDRs to provide a proof-of-concept for low-cost, low power, portable RF systems that can operate on the edge and be used in settings other than a lab environment.

The novel contributions of this paper can be summarized as follows:
\begin{itemize}
  \item Generate a PetaLinux image that can be deployed on the ZYBO SoC Development Board with necessary libraries, packages, and SDR utilities
  \item Create a hardware platform in Vivado that can transfer HackRF I/Q data to and from the PS
  \item Compile a custom interface to be used in the Linux image which can control the HackRF, interface with the hardware platform, and communicate the data over a UDP server
  \item Write an upstream program that can connect to the UDP server on the ZYBO, collect and visualize data, and send commands for controlling the HackRF
\end{itemize}

This paper is organized as follows: Section 2 provides a formally defined problem statement that is specific to the proof-of-concept scope of this project.
Section 3 presents the work done in the project, including block diagrams, photos of the lab setup, and the hardware and software that was created for the project.

\section{Problem Statement}
Most RF systems which include a software defined radio rely on personal computers and accompanying software for controlling the SDR and performing complex signal processing tasks.
However, for applications such as tactical communications or remote sensing, portability and low power is required for these systems to be viable in field testing and deployment.
Therefore, this paper aims to show the replacement of a personal computer with an SoC, which serves as a low power, low cost, and portable alternative with the same computational capability.

The Zybo is a cheap Zynq SoC development board made by Digilent that features Artix FPGA fabric, dual-core ARM A9 processors, 
and different peripherals that can be used for embedded applications. These Zynq SoC’s can run embedded linux with the help of Xilinx’s PetaLinux toolkit, 
allowing for community SDR applications like PySDR and the HackRF API to be leveraged when controlling the SDR. A user can also leverage the FPGA fabric within the 
SoC to create fast parallelized signal processing capabilities that not even a desktop can achieve. 
To this end, I propose S(oC)DR, a Zynq-based framework that abstracts the complexities of interfacing with and 
controlling the HackRF while enabling efficient parallel signal processing.

\nocite{*}
\bibliographystyle{ieeetr}  % Use IEEE format
\bibliography{refs}

\end{document}
