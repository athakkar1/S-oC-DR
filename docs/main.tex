\documentclass[conference]{IEEEtran}
\IEEEoverridecommandlockouts
% The preceding line is only needed to identify funding in the first footnote. If that is unneeded, please comment it out.
%Template version as of 6/27/2024

\usepackage{cite}
\usepackage{amsmath,amssymb,amsfonts}
\usepackage{algorithmic}
\usepackage{graphicx}
\usepackage{textcomp}
\usepackage{xcolor}
\usepackage[hidelinks]{hyperref}
\def\BibTeX{{\rm B\kern-.05em{\sc i\kern-.025em b}\kern-.08em
    T\kern-.1667em\lower.7ex\hbox{E}\kern-.125emX}}
\begin{document}

\title{S(oC)DR: A Low Cost, SoC-Based Software Defined Radio Platform\\
}

\author{\IEEEauthorblockN{Ajay Thakkar}
\IEEEauthorblockA{\textit{Electrical and Computer Engineering} \\
\textit{Stevens Institute of Technology}\\
Hoboken, New Jersey \\
athakka5@stevens.edu \\
Advised by Professor Bernard Yett}
}

\maketitle

\begin{abstract}
Software Defined Radios (SDRs) and System-on-Chips (SoCs) are widely used in modern RF applications due to their flexibility, reconfigurability, and computational efficiency. However, most SDR platforms rely on personal computers for processing, making them unsuitable for portable and low-power applications such as tactical communications and remote sensing. While commercially available SoC-integrated SDR solutions exist, they are often cost-prohibitive, and research-focused implementations are typically built on custom or proprietary hardware, limiting accessibility.

This paper presents S(oC)DR, a low-cost, open-source SoC/SDR development platform designed to enable portable and self-sufficient RF systems. Using the Digilent Zybo Z7-20, a Xilinx Zynq-based SoC development board, and the HackRF One, an open-source SDR, this system demonstrates key functionalities necessary for field-deployable SDR applications. These include SDR parameter tuning, real-time signal processing using programmable logic, and networked data transmission.
  
By integrating SDR control and signal processing within an SoC, S(oC)DR provides a compact, power-efficient alternative to traditional PC-based SDR setups while remaining affordable for researchers and developers. This work aims to bridge the gap between expensive commercial solutions and inaccessible research prototypes, offering a viable platform for advancing portable RF technology.
\end{abstract}

\section{Introduction}
Many modern spectrum tasks require portability and computational power to solve complex signal processing tasks in dynamic environments.
Applications like remote sensing and tactical communications require self sufficient RF systems that can operate on the edge while employing
modern algorithms such as spectrum sensing and signal classification. Creating such devices usually requires
custom hardware and software that are tailored to the application, which is both a costly and non-reusable solution \cite{10880410}. Technology such as
Software Defined Radios (SDR) and System-on-Chips (SoC) can be leveraged in these applications to form fast, reprogrammable, portable, and cost effective systems \cite{9721283}.

SDRs are flexible RF systems that allow for digital reconfigurability to change its operational parameters while running.
Their novelty lies in the interface between a General-Purpose Processor (GPP) and an RF front end, allowing the user to digitally control the front end's
operating frequency, gain, sampling rate, etc. This GPP also allows for a personal computer to communicate with the system to send commands and retrieve return
data for further processing \cite{9721283}. Software such as GNU Radio and MATLAB include packages to communicate with popular SDRs as well as signal processing and visualization
blocks that can be tailored to a wide variety of RF tasks. This SDR ecosystem has been used extensively in modern research for prototyping RF systems,
like Hu et al. utilizing the Pluto SDR platform as well as MATLAB's deep learning toolbox to implement a Convolutional Neural Network (CNN) 
for signal classification of the Pluto SDR's return signal \cite{9687958}. However, since many of these studies use a personal computer for their data processing which have a large form factor 
and utilize a lot of power, these systems are not viable for field testing or deployment. 

To that end, many of these same signal processing applications such as CNNs have been
implemented on programmable logic platforms like SoCs. SoCs contain both Programmable Logic (PL) and a Processing System (PS), essentially 
integrating both FPGA fabric and general purpose processors on the same chip. The processors can execute sequential logic and control
many peripherals that are baked into the chip, such as dedicated UART hardware and GPIO pins. The FPGA fabric executes
parallel logic which can implement tasks such as convolutions or Fast Fourier Transforms much faster than sequential processors.
For example, Amin et al. were able to implement a CNN for signal classification on programmable logic, achieving an accuracy of 99.98\% at 84139 
frames per second with only 4.4W of power consumption, showing how aforementioned SDR research can utilize
SoCs to lower power consumption and increase portability while still retaining the same processing power as a personal computer \cite{10794303}. 
To that end, the focus of this paper is showing a real example of using an SoC with an SDR, and making this technology more accessible to research and commercial applications.

The novel contributions of this paper can be summarized as follows:
\begin{itemize}
  \item Generate a PetaLinux image that can be deployed on the ZYBO SoC Development Board with necessary libraries, packages, and SDR utilities
  \item Create a hardware platform in Vivado that can transfer HackRF I/Q data to and from the PS
  \item Compile a custom interface to be used in the Linux image which can control the HackRF, interface with the hardware platform, and communicate the data over a UDP server
  \item Write an upstream program that can connect to the UDP server on the ZYBO, collect and visualize data, and send commands for controlling the HackRF
\end{itemize}

This paper is organized as follows: Section 2 provides a formally defined problem statement that is specific to the proof-of-concept scope of this project.
Section 3 presents the work done in the project, including block diagrams, photos of the lab setup, and the hardware and software that was created for the project.

\section{Problem Statement}
SoCs and SDRs have already intersected in the commercial and research sectors. In the research realm, compact and low power systems that
leverage software defined radio and SoCs also exist. The CROWN project funded by the European Defense agencies is a 2-18 GHz capable system
that features Xilinx RFSoCs and is labeled as a small-scale, multi-functional RF system. However, due to existing research in this space
frequently using custom hardware for its application or being tied to government funding, this technology is rarely available for purchase
for others to develop and test on \cite{10880410}. In the commercial sector, Ettus Research creates Universal Software Radio Peripheral (USRP) SDRs
which includes an SoC to allow for low-power, edge applications. For example, the USRP E310 is a 70 MHz to 6 GHz software defined radio that 
includes a Xilinx Zynq-based SoC, and is labeled as a "portable stand-alone software defined radio platform designed for field deployment".
While this shows the existance of commercially available SoC/SDR platforms, the cost of these systems range from 5000 all the way up to 20,000 dollars for various
Ettus products, limiting the availability for researchers to prototype and test on \cite{ettus_e310}. This paper endeavours to provide an easily accessible and low cost
SoC/SDR development platform to help progress capabilities in portable, self-sufficient RF systems.

For the scope of this project, the viability of this low cost development platform will be shown through certain key functionalities that existing solutions have.
Most RF applications require the ability to retrieve raw data from the SDR, tune parameters such as gain, center frequency, and sampling rate in real-time, 
implement user defined signal processing chains onto oncoming data, and transmit data over a network \cite{10880410, 9721283}. By integrating these features into an accessible and cost-effective SoC/SDR platform, 
this project aims to lower the barrier to entry for researchers, students, and engineers looking to develop and test self-sufficient and portable RF systems.

\nocite{*}
\bibliographystyle{ieeetr}  % Use IEEE format
\bibliography{refs}

\end{document}
